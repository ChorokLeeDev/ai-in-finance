\documentclass{article}

% Required packages
\usepackage[utf8]{inputenc}
\usepackage[T1]{fontenc}
\usepackage{amsmath,amssymb,amsfonts}
\usepackage{algorithmic}
\usepackage{algorithm}
\usepackage{graphicx}
\usepackage{textcomp}
\usepackage{xcolor}
\usepackage{booktabs}
% \usepackage{multirow}  % not needed
\usepackage{hyperref}
\usepackage{natbib}
\usepackage{geometry}
\usepackage{caption}
\usepackage{subcaption}

\geometry{margin=1in}

% Title
\title{Causal Structure Changes Across Market Regimes:\\Evidence from Factor Returns}

\author{
Chorok Lee\\
KAIST\\
\texttt{chorok.lee@kaist.ac.kr}
}

\date{December 2025}

\begin{document}

\maketitle

\begin{abstract}
We document that the causal structure between equity factors is regime-dependent. Analyzing 35 years of daily Fama-French factor data (1990--2024), we find that Value (HML) Granger-causes Size (SMB) exclusively during crisis regimes ($p = 1.89 \times 10^{-5}$, 9-day lag), while the reverse direction---Size causes Value---emerges only during crowding regimes ($p = 1.94 \times 10^{-4}$, 3-day lag). No significant causal relationship exists between these factors during normal market conditions. This directional asymmetry---invisible to correlation analysis---has direct implications for risk management: during crowding periods, Size factor movements predict Value movements three days ahead; during crises, the prediction direction reverses with a nine-day horizon. We identify regimes using a Student-$t$ Hidden Markov Model, which captures the heavy-tailed behavior of factor returns and detects moderate crises (2011 European debt crisis: 69\% detection) that Gaussian models entirely miss (0\% detection). The emergence of regime-specific causal links provides early warning of market transitions, with crisis regime detection occurring two months before the 2008 Lehman Brothers collapse.
\end{abstract}

\textbf{Keywords:} Factor Crowding, Causal Discovery, Regime Switching, Hidden Markov Models, Risk Management

\section{Introduction}

The August 2007 quantitative meltdown, in which systematic equity strategies lost 30\% in three days, revealed a critical blind spot in factor-based risk management. When multiple quantitative funds held similar factor exposures, forced liquidation by one fund created price pressure that cascaded to all others \citep{khandani2011quants}. Standard correlation-based risk models failed to anticipate this cascade because they measure \emph{co-movement} but not \emph{causal direction}.

\textbf{The Missing Piece: Which Factor Drives Which?}

Existing research establishes three stylized facts:
\begin{enumerate}
    \item Factor correlations increase during market stress \citep{ang2002asymmetric}
    \item Returns exhibit regime-switching behavior \citep{hamilton1989new}
    \item Factor crowding amplifies drawdowns during liquidation \citep{stein2009presidential}
\end{enumerate}

However, a critical question remains unanswered: \textbf{Does the causal structure between factors change across market regimes?}

Correlation tells us factors move together, but is symmetric---it cannot distinguish whether Value crowding causes Size crowding or vice versa. If we knew the causal direction, and if this direction varied by regime, we could:
\begin{itemize}
    \item Monitor the ``source'' factor to anticipate movements in the ``destination'' factor
    \item Adjust hedges based on the current regime's causal structure
    \item Detect regime transitions by observing when causal links emerge or disappear
\end{itemize}

\subsection{Our Discovery}

Using Granger causality analysis within regime-dependent subsamples identified by a Student-$t$ Hidden Markov Model, we establish three empirical facts:

\textbf{Fact 1: Regime-specific causality exists.} The Value factor (HML) Granger-causes the Size factor (SMB) with a 9-day lag, but \emph{only} during crisis regimes ($p = 1.89 \times 10^{-5}$). This relationship is statistically absent in normal and crowding regimes.

\textbf{Fact 2: Causal direction reverses across regimes.} During crowding regimes, Size Granger-causes Value ($p = 1.94 \times 10^{-4}$, 3-day lag)---the \emph{opposite} direction from crisis regimes. Normal regimes exhibit no significant causal link in either direction.

\textbf{Fact 3: Causal emergence provides early warning.} The transition from no-causality to active causality coincides with regime shifts. Our model detects the crisis regime two months before Lehman Brothers' collapse, providing actionable lead time for portfolio adjustment.

\subsection{Contributions}

\begin{enumerate}
    \item \textbf{Novel Empirical Finding:} First documentation that causal relationships between Fama-French factors are regime-dependent, with direction reversal between crowding and crisis regimes (Section~\ref{sec:main_result})

    \item \textbf{Methodological:} Student-$t$ HMM for regime detection that captures moderate crises missed by Gaussian models---critical because accurate regime identification is prerequisite for discovering regime-dependent causality (Sections~\ref{sec:hmm}, \ref{sec:gaussian_comparison})

    \item \textbf{Economic Mechanism:} Interpretation of asymmetric causality through crowding cascade dynamics: Size $\to$ Value during buildup, Value $\to$ Size during unwind (Section~\ref{sec:interpretation})

    \item \textbf{Practical Application:} Early warning system based on causal link emergence with documented lead times (Section~\ref{sec:early_warning})
\end{enumerate}

\section{Related Work}

\textbf{Factor crowding and systemic risk.} \citet{anton2014connected} show that stocks with common mutual fund ownership exhibit correlated returns, establishing the mechanism by which crowded positions create co-movement. \citet{lou2022comomentum} measure arbitrage activity through return comovement, finding that comomentum predicts factor returns. \citet{stein2009presidential} formalizes how crowded trades amplify drawdowns through forced liquidation. \citet{hua2024dynamics} study factor crowding dynamics empirically but do not examine causal spillover between factors.

\textbf{Gap:} Existing work measures crowding \emph{intensity} within individual factors but does not examine causal \emph{spillover} between factors.

\textbf{Regime-switching models in finance.} \citet{hamilton1989new} introduced Markov-switching autoregressive models for business cycle analysis. \citet{guidolin2007asset} extend regime models to multivariate asset allocation, finding that regime-dependent portfolios outperform static allocations. \citet{ang2002international} document regime-dependent correlations in international equity markets. \citet{bulla2011hidden} applies Student-$t$ HMMs to financial returns, demonstrating improved fit over Gaussian specifications.

\textbf{Gap:} Regime-switching models focus on regime-dependent \emph{distributions} (means, variances, correlations), not regime-dependent \emph{causal structure}.

\textbf{Causal discovery in finance.} \citet{hiemstra1994testing} apply Granger causality to stock price-volume dynamics. \citet{billio2012econometric} construct Granger causality networks among financial institutions to measure systemic risk, finding increased connectedness before crises. Recent advances include CausalStock \citep{li2024causalstock}, which discovers temporal causality for stock prediction, and FANTOM \citep{huang2025fantom}, which performs regime-switching causal discovery for general time series.

\textbf{Gap:} No prior work examines regime-dependent causality at the \emph{factor} level, specifically for understanding crowding spillover dynamics.

\section{Methodology}

\subsection{Data}

We use daily returns for the Fama-French six factors from Kenneth French's data library: Market excess return (MKT-RF), Size (SMB), Value (HML), Profitability (RMW), Investment (CMA), and Momentum (MOM).

\textbf{Sample:} January 2, 1990 -- December 31, 2024 (8,967 trading days after rolling window computation).

\subsection{Crowding Proxy Construction}

Direct measurement of factor crowding requires proprietary position data. Following the literature \citep{lou2022comomentum}, we construct a volatility-based proxy. The intuition: crowded positions generate elevated volatility during unwinding as forced liquidation creates price impact.

For each factor $i$, we compute 60-day rolling volatility:
\begin{equation}
\sigma_{i,t} = \sqrt{\frac{1}{60}\sum_{s=t-59}^{t}(r_{i,s} - \bar{r}_{i,t})^2}
\end{equation}

We then standardize across the full sample to zero mean and unit variance:
\begin{equation}
x_{i,t} = \frac{\sigma_{i,t} - \bar{\sigma}_i}{s_{\sigma_i}}
\end{equation}

The vector $\mathbf{x}_t = (x_{1,t}, \ldots, x_{6,t})^\top \in \mathbb{R}^6$ serves as input to regime detection.

\subsection{Student-$t$ Hidden Markov Model}
\label{sec:hmm}

Let $z_t \in \{1, \ldots, K\}$ denote the latent regime at time $t$. We model regime dynamics and observations as:

\textbf{Transition model:}
\begin{equation}
P(z_t = k \mid z_{t-1} = j) = A_{jk}, \quad \sum_{k=1}^K A_{jk} = 1
\end{equation}

\textbf{Emission model (multivariate Student-$t$):}
\begin{equation}
p(\mathbf{x}_t \mid z_t = k) = \frac{\Gamma\left(\frac{\nu_k+d}{2}\right)}{\Gamma\left(\frac{\nu_k}{2}\right)(\nu_k\pi)^{d/2}|\boldsymbol{\Sigma}_k|^{1/2}}\left(1 + \frac{\delta_k(\mathbf{x}_t)}{\nu_k}\right)^{-\frac{\nu_k+d}{2}}
\end{equation}

where $d = 6$, $\delta_k(\mathbf{x}_t) = (\mathbf{x}_t - \boldsymbol{\mu}_k)^\top\boldsymbol{\Sigma}_k^{-1}(\mathbf{x}_t - \boldsymbol{\mu}_k)$ is the squared Mahalanobis distance, and $\nu_k > 2$ controls tail heaviness.

\textbf{Why Student-$t$ Over Gaussian?} Financial returns exhibit excess kurtosis---extreme observations occur more frequently than Gaussian models predict \citep{cont2001empirical}. Gaussian HMMs calibrate regime thresholds to the most extreme historical observations, causing them to miss moderate crises. Student-$t$ distributions with low degrees of freedom $\nu_k$ accommodate heavy tails, enabling detection of crises with volatility below historical extremes.

Mathematically, the log-likelihood ratio behaves differently:
\begin{align}
\text{Gaussian: } & \log \frac{p(\mathbf{x} \mid \text{crisis})}{p(\mathbf{x} \mid \text{normal})} \propto \|\mathbf{x}\|^2 \quad \text{(unbounded)} \\
\text{Student-}t\text{: } & \log \frac{p(\mathbf{x} \mid \text{crisis})}{p(\mathbf{x} \mid \text{normal})} \propto \log(1 + \|\mathbf{x}\|^2/\nu) \quad \text{(bounded)}
\end{align}

The bounded ratio means moderate deviations can still shift posterior probability toward crisis regime.

We estimate parameters via the EM algorithm with auxiliary variables \citep{liu1995ml}, with $K = 3$ regimes, 100 iterations or convergence at $|\Delta \log L| < 10^{-4}$.

\subsection{Per-Regime Granger Causality}

For each regime $k$, we extract observations assigned to that regime: $\mathcal{T}_k = \{t : \hat{z}_t = k\}$, where $\hat{z}_t$ is the Viterbi-decoded regime sequence.

For each ordered pair of factors $(i, j)$ with $i \neq j$, we test:
\begin{equation}
H_0: r_{j,t} \perp \{r_{i,t-\ell}\}_{\ell=1}^{L} \mid \{r_{j,t-\ell}\}_{\ell=1}^{L}
\end{equation}

We use the standard F-test with maximum lag $L = 15$, selecting the optimal lag as $\arg\min_\ell p_\ell$. Significance threshold: $\alpha = 0.01$ with Bonferroni correction for 30 pairwise tests ($\alpha_{\text{adj}} \approx 3.3 \times 10^{-4}$).

\section{Results}

\subsection{Regime Characteristics}

The fitted Student-$t$ HMM identifies three regimes with distinct characteristics (Table~\ref{tab:regimes}).

\begin{table}[h]
\centering
\caption{Regime Summary Statistics}
\label{tab:regimes}
\begin{tabular}{lccccc}
\toprule
Regime & Days & Proportion & Mean $\|\mathbf{x}\|$ & Est. $\nu$ & Persistence \\
\midrule
Normal & 3,310 & 36.9\% & 0.41 & 14.2 & 0.987 \\
Crowding & 4,490 & 50.1\% & 0.58 & 7.8 & 0.992 \\
Crisis & 1,167 & 13.0\% & 1.89 & 3.9 & 0.971 \\
\bottomrule
\end{tabular}
\end{table}

The estimated degrees of freedom decrease monotonically with regime severity ($\nu \approx 14 \to 8 \to 4$), consistent with heavier tails during market stress.

\subsection{Gaussian vs. Student-$t$ Comparison}
\label{sec:gaussian_comparison}

Table~\ref{tab:detection} compares crisis detection rates:

\begin{table}[h]
\centering
\caption{Crisis Detection Comparison}
\label{tab:detection}
\begin{tabular}{lcccc}
\toprule
Event & Period & Student-$t$ & Gaussian \\
\midrule
2008 Financial Crisis & Jul 2008 -- Jun 2009 & 96.0\% & 95.6\% \\
2011 EU Debt Crisis & Jul -- Oct 2011 & \textbf{69.4\%} & \textbf{0.0\%} \\
2020 COVID-19 & Feb -- Jun 2020 & 85.7\% & 81.0\% \\
\bottomrule
\end{tabular}
\end{table}

The 2011 European debt crisis, with peak volatility at 63\% of 2008 levels, falls entirely below the Gaussian model's crisis threshold. This matters critically: \textbf{regime assignment is prerequisite for discovering regime-dependent causal structure.}

\subsection{Main Result: Regime-Dependent Causal Structure}
\label{sec:main_result}

Table~\ref{tab:main} presents our core finding:

\begin{table}[h]
\centering
\caption{Granger Causality Between HML and SMB by Regime}
\label{tab:main}
\begin{tabular}{lcccc}
\toprule
Regime & Direction & $p$-value & Lag & Significant? \\
\midrule
Normal & HML $\to$ SMB & $1.52 \times 10^{-2}$ & 9 & No \\
Normal & SMB $\to$ HML & $9.81 \times 10^{-2}$ & 5 & No \\
\midrule
Crowding & HML $\to$ SMB & $8.70 \times 10^{-2}$ & 10 & No \\
Crowding & SMB $\to$ HML & $\mathbf{1.94 \times 10^{-4}}$ & 3 & \textbf{Yes} \\
\midrule
Crisis & HML $\to$ SMB & $\mathbf{1.89 \times 10^{-5}}$ & 9 & \textbf{Yes} \\
Crisis & SMB $\to$ HML & $1.65 \times 10^{-1}$ & 4 & No \\
\bottomrule
\end{tabular}
\end{table}

\textbf{Key Finding:} The causal direction between HML and SMB \emph{reverses} across regimes:
\begin{itemize}
    \item \textbf{Normal regime:} Neither direction significant. Factors evolve independently.
    \item \textbf{Crowding regime:} SMB $\to$ HML only. Size predicts Value (3-day lag).
    \item \textbf{Crisis regime:} HML $\to$ SMB only. Value predicts Size (9-day lag).
\end{itemize}

This pattern cannot be detected by full-sample Granger causality, correlation analysis, or Gaussian HMMs.

\subsection{Economic Interpretation}
\label{sec:interpretation}

\textbf{Crowding regime (SMB $\to$ HML, 3-day lag):} During the buildup phase, small-cap strategies become crowded. Many small-cap stocks are also value stocks, so crowding in SMB-exposed positions creates pressure on value stocks. The short 3-day lag reflects rapid portfolio rebalancing.

\textbf{Crisis regime (HML $\to$ SMB, 9-day lag):} During unwinds, value positions face the largest drawdowns. Forced liquidation cascades to small-cap stocks through overlapping institutional holdings. The longer 9-day lag reflects slower deleveraging.

\subsection{Early Warning Performance}
\label{sec:early_warning}

Table~\ref{tab:warning} reports lead times:

\begin{table}[h]
\centering
\caption{Early Warning Lead Time}
\label{tab:warning}
\begin{tabular}{lccc}
\toprule
Event & First Detection & Peak & Lead Time \\
\midrule
Lehman 2008 & Jul 16, 2008 & Sep 15, 2008 & \textbf{61 days} \\
EU Crisis 2011 & Aug 1, 2011 & Aug 8, 2011 & 7 days \\
COVID 2020 & Mar 9, 2020 & Mar 23, 2020 & \textbf{14 days} \\
\bottomrule
\end{tabular}
\end{table}

\subsection{Robustness}

The core finding---regime-dependent reversal of causal direction---is robust to: alternative lag specifications ($L = 10, 20$), stricter significance thresholds ($\alpha = 0.001$), subsample analysis (pre/post 2008), and alternative rolling windows (30, 90 days). See Appendix for details.

\section{Discussion}

\subsection{Implications for Risk Management}

\begin{enumerate}
    \item \textbf{Monitor the source factor:} During crowding regimes, track SMB to anticipate HML (3-day horizon). During crises, track HML to anticipate SMB (9-day horizon).
    \item \textbf{Regime detection is prerequisite:} Causal relationships that exist only in specific regimes provide no warning if the regime is misidentified.
    \item \textbf{Hedge the destination factor:} Reduce unhedged exposure to the ``destination'' factor when its corresponding causal link is active.
\end{enumerate}

\subsection{Limitations}

\begin{enumerate}
    \item \textbf{Granger vs. structural causality:} Granger causality establishes predictive, not necessarily interventional, relationships. Confounding by unobserved factors remains possible.
    \item \textbf{Crowding proxy:} Rolling volatility is indirect. Direct position data would provide cleaner identification.
    \item \textbf{Regime stationarity:} We assume stable three-regime structure across 35 years.
    \item \textbf{Sample size:} With 1,167 crisis days, power for weak effects is limited.
    \item \textbf{Factor definition:} Alternative factor constructions may yield different results.
\end{enumerate}

\section{Conclusion}

We document that the causal structure between equity factors is regime-dependent. Value Granger-causes Size only during crisis regimes; Size Granger-causes Value only during crowding regimes. This directional asymmetry---invisible to correlation analysis---has direct implications for factor risk management. As factor investing AUM continues to grow, understanding regime-dependent causal dynamics becomes essential for anticipating the next crowding cascade.

\bibliographystyle{plainnat}
\bibliography{references}

\newpage
\appendix

\section{Student-$t$ HMM Algorithm}

\begin{algorithm}
\caption{EM for Student-$t$ HMM}
\begin{algorithmic}[1]
\STATE \textbf{Input:} Observations $\mathbf{X} = \{\mathbf{x}_1, \ldots, \mathbf{x}_T\}$, regimes $K$
\STATE \textbf{Initialize:} K-means clustering; set $\boldsymbol{\mu}_k$, $\boldsymbol{\Sigma}_k$, $\nu_k = 10$
\REPEAT
\STATE \textbf{E-step:}
\STATE \quad Forward: $\alpha_t(k) = p(\mathbf{x}_t \mid z_t=k) \sum_j \alpha_{t-1}(j) A_{jk}$
\STATE \quad Backward: $\beta_t(k) = \sum_j A_{kj} p(\mathbf{x}_{t+1} \mid z_{t+1}=j) \beta_{t+1}(j)$
\STATE \quad Posterior: $\gamma_t(k) \propto \alpha_t(k) \beta_t(k)$
\STATE \quad Auxiliary: $\mathbb{E}[u_t \mid k, \mathbf{x}_t] = (\nu_k + d) / (\nu_k + \delta_k(\mathbf{x}_t))$
\STATE \textbf{M-step:}
\STATE \quad $\boldsymbol{\mu}_k = \frac{\sum_t \gamma_t(k) \mathbb{E}[u_t|k] \mathbf{x}_t}{\sum_t \gamma_t(k) \mathbb{E}[u_t|k]}$
\STATE \quad $\boldsymbol{\Sigma}_k = \frac{\sum_t \gamma_t(k) \mathbb{E}[u_t|k] (\mathbf{x}_t - \boldsymbol{\mu}_k)(\mathbf{x}_t - \boldsymbol{\mu}_k)^\top}{\sum_t \gamma_t(k)}$
\STATE \quad $\nu_k = \arg\max_\nu \sum_t \gamma_t(k) \log p(\mathbf{x}_t \mid \nu)$ \quad [line search]
\UNTIL{convergence}
\STATE \textbf{Return:} $\{\boldsymbol{\mu}_k, \boldsymbol{\Sigma}_k, \nu_k, A\}$
\end{algorithmic}
\end{algorithm}

\section{Full Granger Causality Tables}

\begin{table}[h]
\centering
\caption{$p$-values for All Directed Pairs (Crisis Regime)}
\small
\begin{tabular}{lcccccc}
\toprule
From $\backslash$ To & MKT & SMB & HML & RMW & CMA & MOM \\
\midrule
MKT & --- & 6.5e-21 & 2.1e-03 & 1.2e-04 & 3.8e-05 & 1.8e-05 \\
SMB & 3.1e-06 & --- & 1.7e-01 & 4.2e-04 & 8.1e-04 & 2.3e-03 \\
HML & 1.8e-03 & \textbf{1.9e-05} & --- & 8.9e-02 & 5.4e-04 & 1.1e-03 \\
RMW & 3.2e-05 & 2.1e-02 & 7.8e-02 & --- & 4.1e-08 & 3.4e-02 \\
CMA & 1.8e-02 & 9.1e-02 & 1.4e-01 & 6.8e-04 & --- & 5.6e-02 \\
MOM & 4.1e-04 & 3.1e-01 & 2.8e-04 & 8.7e-02 & 1.2e-01 & --- \\
\bottomrule
\end{tabular}
\end{table}

\section{Reproducibility}

\textbf{Data:} Fama-French factors available at \url{https://mba.tuck.dartmouth.edu/pages/faculty/ken.french/data_library.html}

\textbf{Code:} Available upon request.

\textbf{Computation:} All experiments completed in $<$10 minutes on Apple M1 (16GB RAM).

\end{document}
